\chapter{Das verwendete Framework}
Zur Umsetzung der Aufgabe muss zun\"achst die Funktionsweise des aktuell verwendeten Verfahrens zur Auswertung und Rekonstruktion eines Bilddatensatzes analysiert werden. Dadurch soll eine Struktur f\"ur die anstehende Implementierung festgelegt werden, die in den existierenden Programmcode nur minimal eingreift. So muss der Code dahingehend modifiziert werden, dass die normale Funktionsweise nicht beeintr\"achtigt wird und das Programm bei Bedarf auch ohne die geplante Erweiterung funktionsf\"ahig bleibt.


\section{Multi-View Environment}
Das Multi-View Environment (MVE)\cite{fuhrmann2014mve} dient zur geometrische Oberfl\"achenrekonstruktion aus zuvor erfassten Bilddaten. Es implementiert eine end-to-end Pipeline dessen einzelne Abschnitte individuell ansprechbar sind. Fundament f\"ur jede Anwendung ist ein Datensatz an Fotografien eines Objektes, welche aus ausreichend vielen Winkeln akquiriert sein m\"ussen, damit eine l\"uckenlose 3D-Rekonstruktion erfolgen kann. Im ersten Abschnitt Structure-from-Motion (SfM) werden Bilder zu einander registriert indem jedes Foto eine Merkmalsextraktion durchl\"auft, die anschlie\ss end dazu dienen die eigentlichen Kamerapositionen zur\"uckzurechnen. Bilder werden samt ihrer detektierten Merkmale (engl.: Features) und anderer Parameter als Views zusammengefasst. Daraufhin ist es mittels Triangulation m\"oglich, \"uber korrespondierende Bildbereiche auf Tiefenwerte zu schlie\ss en und zweidimensionale Bildpunkte (x,y) in ein dreidimensionales Koordinatensystem (x,y,z) zu \"ubertragen, was den Vorgang des Multi-View Stereo (MVS) Abschnitts beschreibt. Als letztes erfolgt die Oberfl\"achentexturierung, die sich aus den Eingabebildern ergibt.

\subsection{Qt und die Benutzeroberfl\"ache}
Die einzelnen Schritte der MVE-Pipeline sind via Kommandozeile zug\"anglich, sodass zur leichteren Zug\"anglichkeit f\"ur Nutzer eine hauseigene, auf der Qt\footnote{\url{http://qt-project.org}} Bibliothek basierende Anwendung \glqq Ultimate MVE\grqq \ (UMVE) entworfen wurde. Jene soll in dieser Arbeit als Grundlage genutzt und um Funktionalit\"aten erweitert werden.\\
Die plattform\"ubergreifende C++ Klassenbibliothek Qt 4.0 bietet, anstelle des zuvor simuliertem Aussehens verschiedener Plattformen, native Schnittstellen an, um betriebssystemeigene Routinen zum Zeichnen der Oberfl\"achenelemente zu nutzen. Somit soll im Sinne der Aufgabenstellung eine Nutzerschnittstelle entworfen werden, die es erm\"oglicht einzelne Bildmerkmale mit inhaltlichem Bildkontext in Relation zu setzen. Die M\"oglichkeit Fotografien vorab manuell zu unterteilen erscheint notwendig um eine Relation zwischen Bildpunkten und ihrer zugeh\"origen Geb\"audeseite zu erstellen. Eine Anforderungsanalyse an die Nutzerschnittstelle ergibt ein Minimum an Funktionalit\"aten, die durch die Implementierung gew\"ahrleistet sein m\"ussen - die M\"oglichkeit Bildbereiche zu markieren, zu editieren und f\"ur das weitere Vorgehen verf\"ugbar zu machen. 


\subsection{Plugin-Integration}

Die zuvor genannte Qt Bibliothek soll verwendet werden um f\"ur die bereits bestehende Software eine Erweiterung zu implementieren. Die M\"oglichkeit die zugrundeliegende Software um Funktionalit\"aten zu erweitern, ist im Vergleich zu einem individuellen Programm vorteilhafter. Durch Implementierung einer externen Software entst\"unde die Notwendigkeit des Ein- und Auslesens notwendiger Daten. Damit die in MVE verwendeten Views nicht importiert und Ergebnisse des sp\"ater genauer beschriebenen Verfahrens nicht exportiert werden m\"ussen soll dieser Schritt in die MVE Pipeline integriert werden. Daher wurde sich auf eine modale Umsetzung in Form eines Plugins geeinigt, welches sich sp\"ater leicht in UMVE integrieren lassen soll. Zu den bereits bestehenden Darstellungen von UMVE kommt ein zus\"atzlicher Reiter hinzu, der trotz der Eingliederung die neuen Funktionalit\"aten getrennt anzeigt. Der Aufbau des geplanten Plugins orientiert sich dabei an bereits existierenden Ansichten der Software, sodass f\"ur den Nutzer eine gewohnte Arbeitsumgebung erhalten bleibt. 